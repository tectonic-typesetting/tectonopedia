\explainer{what-is-tex}{What is \TeX?}{What is TeX?}

The Tectonic project is founded on a technology called \TeX. What is \TeX, and
why does it matter?

The best way we can put it is that \i{\TeX\ is a programming language for
creating documents}. \TeX\ doesn't look like most programming languages, because
most of the time you're just typing the words that you want to appear. But when
you want to put some text \i{in italics}, or conjure up an equation like $f(x) =
x^3 - \beta$, or start a new chapter in your book, there are special \TeX\
commands to type that cause those things to happen. You don’t see this
terminology too often anymore, but \TeX\ is the antithesis of a “WYSIWYG” —
“what you see is what you get” — tool like
\href{https://www.microsoft.com/en-us/microsoft-365/word}{Microsoft Word}.

This article is \i{not}, however, going to introduce any specifics of the \TeX\
programming language. Instead, it aims present some of the non-technical aspects
of \TeX\ that we think are valuable to understand even if you never end up
delving deep into its guts. \TeX\ is arguably a unique piece of technology in
the landscape of 21st-century computing, and we hope that familiarity with some
of its background will help you better understand the distinctive
characteristics of Tectonic and other \TeX-based software.

It would be very reasonable to start by asking, \i{why the heck would I want to
use a programming language to write a report anyway?} A programming language
sounds like it could be \i{powerful}, but that's not a word that necessarily
comes to mind in relation to writing tools. And, indeed, there are plenty of
jobs that can be done just as well with a tool like
\href{https://en.wikipedia.org/wiki/Markdown}{Markdown} instead of \TeX.

Some kinds of documents really do benefit from power, though. In particular,
tools like Markdown give you a fixed set of building blocks to use: a heading
here, a bulleted list there. If the document you want to create uses only those
building blocks, great. The unique power of \TeX\ is that it allows you to
create and use \i{new} building blocks as you need them. This flexibility is
really important for typesetting mathematics, which is probably \TeX's most
famous strength, but it applies in all kinds of situations.

And we'll further claim that \i{once you stop accepting the limitations that
less powerful tools impose on you, you'll start seeing opportunities to use
\TeX's capabilities everywhere}. That being said, most people do find that \TeX\
makes the biggest difference for \i{technical documents}: the sorts of works
that contain math, figures, tables, code, or lots of cross-references. \TeX\ was
primarily designed for these use cases.

% FIXME: without the hspace, the canvas-start special seemingly has a weird
% position that causes the word TeX to appear in a bad position. The
% canvas-processing seems OK; it's just that the reference point for all of the
% letter positions is incorrect!
\hspace{0in}\TeX\ has a lot of history behind it. The first version of \TeX\ was created
by legendary computer scientist
\href{https://en.wikipedia.org/wiki/Donald_Knuth}{Donald E.\ Knuth} and released
all the way back in 1978. Knuth had been disappointed by the quality of the
typesetting of one of his books and, after seeing some examples of what
modern-circa-1977 digital printing systems could produce, decided that he could
write a computer program that would help him meet a higher standard. The
software that he created was a remarkable, visionary achievement. It certainly
succeeded in its core aim: \TeX\ set a new standard in terms of the sheer
quality of its typesetting outputs. In order to achieve this result, Knuth
created the first high-quality digital fonts and developed pioneering new
algorithms for laying out mathematics, breaking paragraphs into lines, and
automatically hyphenating words (a capability needed to produce acceptable
fully-justified text). \TeX\ was \i{also} famous for the quality of its
engineering: it was efficient, reliable, ultra-precise in its output, and as
close to bug-free as software can get. Finally, Knuth made \TeX\ freely
available, with its implementation open for inspection and customization.

To frame it another way: there are very few pieces of software that were
developed before 1980 that are still in everyday use. \TeX\ is one of them.

Now is as good a time as any to mention that the name “\TeX” is pronounced
“tech”, similar to the word “loch”. It's meant to evoke the Greek root of the
word “technology”, with the “X” representing the Greek letter chi, χ. (This
might seem like a bit of an odd choice since one of \TeX's exciting capabilities
was that it could actually display Greek letters, which are routinely needed in
equations!) The preferred way to write the name of the software involves the
lowered (“kerned”) “E”, which is another thing that few systems at the time
could do. In situations where you can't control kerning, the usual way to write
the name of the software is “TeX”.

It's also worth pointing out that we've been using one word, “\TeX”, to refer to
two different things. \TeX, a programming language, is processed by \TeX, a
specific computer program. In principle, different programs might implement the
\TeX\ programming language.

In fact, modern users almost always use extended variants of the \TeX\ program,
which have names such as \texttt{pdftex}, \texttt{xetex}, and Tectonic. These
variants add new features to Knuth's \TeX\ language such as support for more
modern font formats. These variants are developed in the context of Knuth's firm
wishes that the name “\TeX” be reserved for programs that are thoroughly
(\i{really} thoroughly) compatible with his. People will sometimes use the word
“\TeX” to refer to this broader family when the intent is clear, as these
related programs have much more in common than they differ. This documentation
follows that pattern, sometimes using the term “Knuth \TeX” to refer
emphatically to the non-extended version of the language.

People will also use the word “\TeX” to refer to the broader ecosystem of
software needed to create finished documents using the \TeX\ language —
nowadays, doing this requires a whole suite of programs and data that all work
together, with the core \TeX\ “engine” being only one of those pieces. For
instance, the majority of modern \TeX\ documents are written using the \LaTeX\
system, which extends the baseline \TeX\ language with higher-level constructs
amounting to tens of thousands of lines of code. \LaTeX\ can itself be extended
with \i{thousands} of add-on packages. The term “\TeX” is often used as
\href{https://en.wikipedia.org/wiki/Synecdoche}{synecdoche} for this broader
ecosystem.


\section*{See Also}

\begin{itemize}
  \item \pediaLinkRef{explainers}{why-tex}
\end{itemize}