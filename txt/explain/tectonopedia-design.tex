\Explainer{tectonopedia-design}{The Design of the Tectonopedia}{The Design of the Tectonopedia}

At its heart, the Tectonopedia is a technical documentation website created by a
software tool. While there are many existing technical documentation systems,
there are \i{even more} choices that can go into their design, and Tectonopedia
occupies a somewhat unusal niche. This page attempts to explain some of the key
decisions and their motivations.

\section*{\TeX-Based}

The biggest decision of all is that the underlying content of the Tectonopedia
is expressed as \TeX\ source. While it's surely possible to construct a similar
system based on formats like \`Markdown`, the whole animating principle of the
Tectonic project is that \TeX\ is a uniquely powerful tool for creating complex
technical documentation. So, Tectonopedia aims to demonstrate that. The article
\explain{why-tex} explores this topic in greater depth.

\section*{Bundler-Powered}

The approach taken by Tectonic and Tectonopedia is unique, as far as we know, in
that the HTML output produced from the \TeX\ sources is intended to be passed
through a modern web bundler framework. In the specific case of Tectonopedia,
\href{https://yarnpkg.com/}{Yarn} is used to drive the
\href{https://parceljs.org/}{Parcel.js} bundler, and the frontend of the website
is a \href{https://vuejs.org/}{Vue} app written in
\href{https://typescriptlang.org/}{TypeScript}. (These particular tools were
chosen because of the qualitative feeling that they are better-engineered than
their competitors.) While the Tectonopedia software can emit a full tree of HTML
files without requiring these external tools, the final assembly of the website
relies on them.

This design is motivated by the belief that featureful, modern web apps are
simply too complex to be created in any other way. While present-day
JavaScript/web development ecosystem has a \i{lot} of frustrating
characteristics, it does enable people to create apps that are much more
sophisticated than can be achieved with hand-coded HTML, CSS, and JavaScript.
And something like Tectonopedia is best thought of as a web application that
just happens to be very text-heavy.

Although Tectonic does not yet support this, in the future its output might
become even more bundler-centric by support component markup languages like
\href{https://react.dev/learn/writing-markup-with-jsx}{JSX} as well as plain
HTML. This would provide a clean path for a \TeX\ document to include, say,
interactive buttons and controls, not just the kinds of content that can easily
be created with standard HTML tags.

\section*{Like Wikipedia, Not a Linear Book}

As the name might suggest, the content structure of Tectonopedia is \i{strongly}
inspired by \href{https://www.wikipedia.org/}{Wikipedia}: it’s a big bag of
content with no overarching structure imposed. This structure is in contrast to
that demanded by tools like \href{https://rust-lang.github.io/mdBook/}{mdBook}
or, well, regular printed books, which require that all of their content be
placed into an essentially linear structure.

The linear structure imposed by the book format becomes unwieldy for reference
documentation when there are lots of small items (e.g., individual \TeX\
commands) that each need to be treated. The book format adds a barrier to
including new content in the collection: one must decide where to slot it into
the global linear structure. Sometimes this decision is easy, but sometimes it
can be quite challenging, and it is unfriendly to the sort of distributed
authorship that has allowed Wikipedia to reach its uniquely expansive scope.

The unstructured “encyclopedia” format has its own risk: content can simply get
lost if nothing links to it. Preventing this is in fact a major Tectonopedia
design goal. Fortunately, one of the advantages of \TeX\ is that it can make
rich internal cross-referencing extremely easy, and full-text search was
prioritized as an early feature of the Tectonopedia impementation.

\section*{Rich Cross-Referencing}

The remark at the end of the previous section bears emphasis. One of the core
tenets of the Tectonopedia design is that rich cross-referencing is an
absolutely fundamental characteristic of technical documentation, and that the
Tectonopedia software should therefore make the creation of cross-references as
easy as possible.

If you think about most API documentation tools like
\href{https://www.sphinx-doc.org/}{Sphinx} or
\href{https://doc.rust-lang.org/rustdoc/}{Rustdoc}, cross-referencing is
generally an essential feature. However, the documentation created by these
tools is generally self-contained: sophisticated cross-referencing becomes much
more difficult if you wish to leave the confines of the language being
documented, and linking \i{into} the documentation is often difficult. An
animating principle of Tectonic and the Tectonopedia is that the power of the
\TeX\ language should make it possible to cross-reference richly across
languages and frameworks.

\section*{Not Just Reference Material}

The content design of Tectonopedia is heavily influenced by the \`Divio
documentation system`, which argues that technical documentation can be broken
into four major categories: reference material, how-tos, tutorials, and
explainers. Many documentation collections \i{heavily} overweight reference
material at the expense of the other categories.

Traditional “encyclopedia-style” content corresponds closely to the “reference
material” category of the Divio model. Other content categories, like tutorials
and how-tos, could potentially be hosted in another framework. But why not keep
it all in one place? A unified structure will hopefully encourage more balance
among the different categories, as well as once again promoting the rich
cross-referencing that we seek.

\section*{Git-Backed}

Tectonopedia content is stored in a Git repository, as opposed to other storage
systems such as the relational databases used by many wiki and
content-management systems.

While there's nothing wrong about using a relational database for content
management, a Git-based approach fits well with the rest of the Tectonopedia
design as it currently stands: the Tectonopedia framework spans a Rust program,
\TeX\ support files, TypeScript/Vue frontend code, and the actual encyclopedia
content.

This design may pose a challenge in the future because we hope that one day
Tectonopedia will be editable directly through its web interface, in the style
of traditional wikis. The technical implementation of this capability seems
challenging, although there are plenty of Git-backed wikis out there like
\href{https://github.com/gollum/gollum}{gollum}.

\section*{Implicit URL Structure}

Many file-based site generators adopt a simple scheme where the files that you
store in your repository map directly to their public URLs: the text in
\texttt{about/our-story.md} is exposed at the URL \texttt{/about/our-story/},
and so on. We call this an \i{explicit} URL structure. Tectonopedia takes a
different, \i{implicit} approach.

The explicit URL approach does not mesh well with Tectonic's encyclopedia-style,
big-bag-of-content scheme, which encourages a flat URL hierarchy. Consider
Wikipedia itself, where millions of pages may live directly under the
\texttt{/wiki/} path prefix. It would be extremely inconvenient to have to
mirror this structure inside one's Git repository!

Instead, Tectonopedia's URL structure is implicit in its content. An input file
may contain a command \tex`\Explainer{tectonopedia-design}` that declares a new
explainer living at the URL path \texttt{/explain/tectonopedia-design/},
regardless of the path of the actual input file. In fact, in Tectonopedia, one
input file can declare any number of outputs (zero, one, or many), allowing
several similar pages to be managed in a single file.

If the file structure of the inputs doesn't affect the URL structure of the
outputs, what \i{does} it affect — how should the input filenames be structured?
In a certain sense, it just doesn't matter. But, those are the files that you
actually open up and edit to update the Tectonopedia content, and in the future,
the aspiration is that the Tectonopedia code will analyze the Git history of its
backing repository to do things like assign authorship and expose the
modification history of various pages. So the structure of input files should
really be whatever makes editing convenient, including the knowledge that
changes will be tracked in Git and eventually surfaced in the user interface.

Another yet-to-be-implemented feature is logging of the exposed Tectonopedia URL
structure such the tool can make sure that existing URLs don't break even as
users rename pages. The public Tectonopedia URL structure can be thought of as
an “API” that we want to avoid breaking whenever possible.
