\explainer{why-tex}{Why \TeX?}{Why TeX?}

Technology has come a long way since the 1970's. Why is Tectonic encouraging
people to create documents using the venerable \TeX\ language, which was
designed at a time when computers — and computing — were so different than they
are today?

First things first: we'll happily admit that there are plenty of circumstances
where \TeX\ and Tectonic are \i{not} the best solutions, and you'll be better off
using some other kind of technology — whether that's
\href{https://en.wikipedia.org/wiki/Markdown}{Markdown},
\href{https://www.microsoft.com/en-us/microsoft-365/word}{Microsoft Word}, pen
and paper, or whatever else. The very notion of “creating documents” is so broad
that it should go without saying that no single system is going to be the best
choice for every situation.

That being said, Tectonic is inspired by the belief that despite its age \TeX\
is still the very best tool in the world for solving certain kinds of problems.
For instance, if you know one thing about \TeX, it's probably that it's good for
mathematics. And that reputation is well-earned! A proficient \TeX\ user can
easily write a single line of code to conjure up this equation:
\[
\int\limits_{-\infty}^{\infty} \text{d}x \, e^{-x^{2}} = \sqrt{\pi}.
\]
What should actually impress you more, however, are equations typeset inline
with body text, like $y = x^2$. Readability requires that the placement, sizing,
and appearance of the math and text symbols all agree well, issues that can be
fudged a bit in “display” equations like the one above. Here too, \TeX\ is
best-in-class.

But we wouldn't be writing all this verbiage if \TeX\ just had a nice math
layout algorithm. \i{Why} is \TeX\ so much better at typesetting math than other
systems? We claim that the crux of the matter is that written mathematics is an
open-ended visual language: it admits infinitely varied forms in unique,
unpredictable combinations. \TeX\ can handle math well not because it got some
specific fiddly bits right — although it did — but because \i{\TeX\ is itself an
expressive, open-ended language} capable of producing forms that weren't
anticipated when it was designed.

If we interpret the title of this article to be asking the question \i{why not
just use Markdown?}, this is our answer. We can think of \TeX\ as a \i{document
language} that is fundamentally more powerful than fixed grammars like Markdown.
(We're being handwavey with terminology here, but for what it's worth, the \TeX\
language is
\href{https://en.wikipedia.org/wiki/Turing_completeness}{Turing-complete}.)
Tools like Markdown may provide grammars for creating documents, but they don't
allow you to establish new grammars of your own. This is far from the only
reason that \TeX\ is good at what it does, but it's the most essential one.

Now, anyone with a healthy skepticism of abstraction is likely to respond: \i{I
don't want some elaborate system that says it can do everything — I just want
something to get the job done}. This is the right response! Human lifetimes have
been wasted on the refinement of elegant but useless ideas, and we have
deadlines to meet. But hopefully that you'll agree that sometimes a system of
abstraction is exactly what you need to get the job done: try doing physics
without calculus.

If that's the case, the question is really about what kind of balance can be
struck: between power and simplicity, flexibility and ease. This relates to the
narrower issue of why Tectonic specifically promotes \TeX\ and not some
hypothetical \i{other} language for creating documents. The claim that we make,
but can't possibly prove, is that \TeX\ strikes this balance amazingly well.
When you just need to ``get the job done,'' it can get out of your way. But if
you need to create a new abstraction, it can do that too.

A great example of this is the
\href{https://en.wikipedia.org/wiki/BibTeX}{Bib\TeX} system for dealing with
bibliographic references in \TeX\ documents. Although the core \TeX\ language
doesn't know anything about bibliographic references, it's flexible enough that
Bib\TeX\ is able to extend it with new commands to support them. And that
support doesn't feel like an awkward extension: the new commands integrate
straightforwardly with the rest of the language and are intuitive for users,
even ones who don't care about how Bib\TeX\ works under the hood. Most people
who have gotten the hang of Bib\TeX\ would probably \i{hate} the idea of giving
it up and going back to managing their references manually.

There are other reasons to embrace \TeX. It may be old, but by the same token it
is \i{battle-tested} and amazingly reliable. It is \i{fast} and ingeniously
efficient. And it is \i{supported} by a worldwide community of users, who have
gone to incredible lengths to modernize it and develop a dizzying array of
extension packages. A lot of very smart people have put a lot of effort into
this language, which is still going strong after forty years — and those facts
tell you something important.

That is not to imply that today's \TeX\ is perfect — far from it. The error
messages are famously hard to understand. Its documentation is, ironically, a
mess. Indeed, a major premise of the Tectonic project is that some aspects of
the \TeX\ ecosystem are in need of dramatic change. But not all of them.
Tectonic is founded on the idea that \TeX\ \i{can be} the document language that
the 21st century deserves.
