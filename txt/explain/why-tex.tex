\Explainer{why-tex}{Why \TeX?}{Why TeX?}

Technology has come a long way since the 1970's. Why is Tectonic encouraging
people to create documents using the venerable \TeX\ language, which was
designed at a time when computers — and computing — were so different than they
are today?

First things first: we'll happily admit that there are plenty of circumstances
where \TeX\ and Tectonic are \i{not} the best solutions, and you'll be better off
using some other kind of technology — whether that's
\href{https://en.wikipedia.org/wiki/Markdown}{Markdown},
\href{https://www.microsoft.com/en-us/microsoft-365/word}{Microsoft Word}, pen
and paper, or whatever else. The very notion of “creating documents” is so broad
that it should go without saying that no single system is going to be the best
choice for every situation.

That being said, Tectonic is inspired by the belief that despite its age \TeX\
is still the very best tool in the world for solving certain kinds of problems.
For instance, if you know one thing about \TeX, it's probably that it's good for
mathematics. And that reputation is well-earned! A proficient \TeX\ user can
easily write a single line of code to conjure up this equation:
\[
\int\limits_{-\infty}^{\infty} \text{d}x \, e^{-x^{2}} = \sqrt{\pi}.
\]
What should actually impress you more, however, are equations typeset inline
with body text, like $y = x^2$. Readability requires that the placement, sizing,
and appearance of the math and text symbols all agree well, issues that can be
fudged a bit in “display” equations like the one above. \TeX\ is one of the few
tools out there that can get all these things just right.

But we wouldn't be writing all this verbiage in honor of a finicky math layout
algorithm. Why can't other tools just copy \TeX's algorithms and do math equally
well? We claim that the real challenge of typesetting mathematics is that
written math is an open-ended visual language, admitting infinitely varied forms
in unique, unpredictable combinations. \TeX\ can handle math well not because it
got some specific fiddly bits right — although it did — but because \i{\TeX\ is
itself an expressive, open-ended language}. Other tools give you building
blocks; only \TeX\ gives you the machinery to create and use new blocks of your
own. (This is not an equivalent statement, but for what it’s worth, the \TeX\
language is
\href{https://en.wikipedia.org/wiki/Turing_completeness}{Turing-complete}.) And
your own “blocks” can be just as easy and natural to use as the built-in ones.
This is far from the only reason that \TeX\ is good at what it does, but it's
the most essential one.

The power of \TeX\ manifests itself not only in low-level ways, such as math
typesetting, but in higher-level ones as well. For instance, scholarly documents
have detailed conventions for handling bibliographic references. Although
neither the core \TeX\ language nor Microsoft Word have any built-in, structured
way to represent reference metadata, \TeX\ has been extended to support them in
the \href{https://en.wikipedia.org/wiki/BibTeX}{Bib\TeX} framework. Bib\TeX's
new commands don't feel like awkward extensions: they integrate
straightforwardly with the rest of the language and are intuitive for users.
While it's true that you can manually typeset your references in Word and
assemble them into a bibliography, it's fair to say Bib\TeX\ provides a
fundamentally more powerful way to work with them: we’ll wager that most people
who have gotten the hang of Bib\TeX\ would \i{hate} the idea of giving it up and
going back to managing their references manually.

Now, if you’re a healthy skeptic of abstraction, you'll likely respond: \i{whoa,
I don't want some elaborate system that can do anything — I just want a tool
that helps me get the job done}. This is the right response! Human lifetimes
have been wasted on the refinement of elegant but useless ideas, and we have
deadlines to meet. But hopefully that you'll agree that in some situations, a
system of abstraction is exactly what you need to get the job done. Try doing
physics without calculus.

We'll assert, but can't possibly prove, that \i{once you stop accepting the
limitations that less powerful tools impose on you, you'll start seeing
opportunities to use \TeX's capabilities everywhere}. Many kinds of documents
have a sort of “internal logic” that becomes easier to express given the right
tools. That being said, the ones where \TeX's capabilities generally add the
most are ones where this internal logic is easy to find: documents with lots of
cross-references, figures, tables, equations, or code — what the Tectonic
community calls \i{technical documents}. And it is likely no coincidence that
these are the sorts of documents that \TeX\ has historically most often been
used to create.

Further, we’ll boldly claim that despite its internal sophistication, \TeX\ is
\i{easy to start using}. We can't deny that \TeX\ has a reputation for confusing
output and sometimes inexplicable behavior, or that there are reasons that this
reputation is deserved. Nevertheless, we'll point out that many mathematicians
and scientists who do not care \i{at all} about its guts successfully use it for
everyday work, even if it drives them off the wall at times. You can think of it
as being a bit like \href{https://git-scm.com/}{git} in this way.

There are other reasons to embrace \TeX. It may be old, but by the same token it
is \i{battle-tested} and amazingly reliable. It is \i{fast} and ingeniously
efficient. It is \i{supported} by a worldwide community of users, who have gone
to incredible lengths to modernize it and develop a dizzying array of extension
packages. A lot of very smart people have put a lot of effort into this
language, which is still going strong after forty years — and those facts tell
you something important.

That is not to imply that today's \TeX\ is perfect — far from it. The error
messages are famously hard to understand. Its documentation is, ironically, a
mess. Indeed, a major premise of the Tectonic project is that some aspects of
the \TeX\ ecosystem are in need of dramatic change. But not all of them.
Tectonic is founded on the idea that \TeX\ \i{can be} the document language that
the 21st century deserves.

\section*{See Also}

\begin{itemize}
  \item \pediaLinkRef{explainers}{what-is-tex}
\end{itemize}