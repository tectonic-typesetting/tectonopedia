\entry{markdown}{Markdown}{Markdown}
\DeclareTerm{Markdown}

Markdown is a plain-text markup format that can be converted into HTML. It’s
very popular.

The article \explain{why-tex} attempts to motivate the reasons to keep on using
the \TeX\ language in a world where Markdown is ubiquitous. That is not to claim
that \TeX\ is somehow \i{better} than Markdown (or any other similar language) —
each system has its own strengths and weaknesses. While it is fair to say that
\TeX\ is more \i{powerful} than Markdown, it is also fair to say that Markdown
is \i{simpler} than \TeX.

\section*{See Also}

\begin{itemize}
\item \href{https://en.wikipedia.org/wiki/Markdown}{The English Wikipedia page on Markdown}
\item \href{https://commonmark.org/}{CommonMark}, “strongly defined, highly
  compatible specification of Markdown”
\item \explain{why-tex} article
\end{itemize}
