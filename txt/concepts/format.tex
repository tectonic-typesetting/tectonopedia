\entry{format}{Format}{Format}

A \TeX\ \b{format} is a predefined set of macros and settings that is loaded
automatically by the engine upon startup, before performing any processing of
the input document. Different formats generally have well-known names, the most
famous of which is \LaTeX.

Tectonic can generate formats on-demand. A format named \texttt{EXAMPLE} is
created by processing an input file named \texttt{tectonic-format-EXAMPLE.tex};
for instance, the default \LaTeX\ format is set up by processing
\texttt{tectonic-format-latex.tex}. This input is processed by the engine in
“initex” mode, which is a special engine mode that finishes by emitting a
\e{format-file} rather than the usual XDV file.

\section*{See Also}

\begin{itemize}
\item \e{format-file}
\item \e{dump}, the command that terminates processing of a format input file
\end{itemize}

\tduxEmit


\entry{format-file}{Format file}{Format file}

A \b{format file} is a binary file encoding a \TeX\ \e{format}.

Each \e{format} is saved into, and loaded from, a format file. In Tectonic,
format files are automatically generated on demand and saved in the user cache
directory. The contents of a format file are closely tied to the internal layout
of the engine's data structures, and format files will generally need to be
regenerated when the engine version changes. In Tectonic, the engine is
versioned with a “format serial” number that is incremented every time something
changes that will require format files to be regenerated.

\section*{See Also}

\begin{itemize}
\item \e{format}
\end{itemize}
