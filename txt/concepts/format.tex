\Entry{format}{Format}{Format}
\DeclareTerm{format}

A \TeX\ \b{format} is a predefined set of macros and settings that is loaded
automatically by the engine upon startup, before performing any processing of
the input document. Different formats generally have well-known names, the most
famous of which is \LaTeX.

Formats are created by running the \TeX\ engine in a special “initex” mode. In
this mode, the engine starts up with a built-in initial configuration and then
processes its input more-or-less as normal. However, processing is terminated
with \e{dump} rather than \`the @Bend primitive`, and at that point the engine
emits a special \`format file` rather than the usual XDV file.

Tectonic manages format files specially, and can generate them on-demand. If an
input document requires a format named \texttt{EXAMPLE}, Tectonic will seek to
generate its format file by processing an input file named
\texttt{tectonic-format-EXAMPLE.tex}. For instance, the default \LaTeX\ format
is set up by processing \texttt{tectonic-format-latex.tex}. The format file is
then saved in the user's cache directory.

\section*{See Also}

\begin{itemize}
\item \e{format-file}
\item \e{dump}, the command that terminates processing of a format input file
\end{itemize}


\Entry{format-file}{Format file}{Format file}
\DeclareTerm{format file}
\DeclareTerm{format files}

A \b{format file} is a binary file encoding a \TeX\ \`format`.

Each \`format` is saved into, and loaded from, a format file. In Tectonic,
format files are automatically generated on demand and saved in the user cache
directory. The contents of a format file are closely tied to the internal layout
of the engine's data structures, and format files will generally need to be
regenerated when the engine version changes. In Tectonic, the engine is
versioned with a “format serial” number that is incremented every time something
changes that will require format files to be regenerated.

\section*{See Also}

\begin{itemize}
\item \e{format}
\end{itemize}
