\entry{divio-documentation-system}{Divio documentation system}{Divio documentation system}
\DeclareTerm{Divio documentation system}

The \b{Divio documentation system} is a framework for organizing technical
documentation. It is described at
\href{https://documentation.divio.com/}{documentation.divio.com}, which refers
to it simply as “the documentation system”. For clarity, we give it a more
specific name. The structure of the Tectonopedia framework is strongly inspired
by the Divio system.


\section*{Framework}

The basic premise of the Divio documentation system is that technical
documentation can be divided into four categories: \i{tutorials}, \i{how-to
guides}, \i{reference material}, and \i{explainers}. These can be understood as
occupying four quadrants in a space defined by two axes: \i{practicality} and
\i{goal-orientation}.

\begin{enumerate}
  \item \b{Tutorials} are practical but not goal-oriented. They are intended for
  learners who are interested in a tool, but don't know much about it; they
  don't even know what they need to know. They introduce novices to various
  aspects of a tool in a highly scripted way, assuming minimal levels of
  previous familiarity. The Divio website describes tutorials as
  \i{learning-oriented}.

  \item \b{How-tos} are practical and goal-oriented. They are intended for
  people who have a problem, and \i{know} that they have a problem, but don't
  know how to solve it. The ideal how-to provides a straightforward recipe for
  solving the problem in question. The Divio website describes how-tos as
  \i{problem-oriented}.

  \item \b{Reference materials} are goal-oriented but abstract. They provide
  detailed technical information about a particular concept or aspect of a
  technology system. For example, the encyclopedia-style entries in the
  Tectonopedia fit this description.  A user that needs to solve a problem
  relating to a particular aspect of a tool should be able to find the
  information they need in its reference, but the reference should not attempt
  to foresee particular user problems or goals. The Divio website describes
  reference materials as \i{information-oriented}.

  \item \b{Explainers} are abstract and not goal-oriented. They provide
  high-level explanations of the \i{hows} and \i{whys} of the design of a tool
  or technology. The Divio website describes explainers as
  \i{understanding-oriented}.
\end{enumerate}


\section*{Limitations}

The Divio system doesn't cover all possible documentation needs and styles. Some
limitations include:

\begin{enumerate}
  \item It does not discuss \b{landing pages}, \b{cheat sheets}, \b{quick
  references} or other \href{https://en.wikipedia.org/wiki/Finding_aid}{finding
  aids}. While full-text search is probably the most important finding aid that
  can be provided, it is valuable to have documentation pages that efficiently
  route readers to the right piece of documentation. Once a documentation corpus
  reaches a sufficient size, a “one-size-fits-all” landing page becomes
  unwieldy, and finding aids that are tuned to particular user profiles become
  helpful. Likewise, it is valuable to have documentation pages that provide
  comprehensive but terse indices of, say, all of the commands that are
  available in a particular program mode.
\end{enumerate}