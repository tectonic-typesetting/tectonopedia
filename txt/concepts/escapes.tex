\Entry{tex-special-characters}{\TeX\ Special Characters}{TeX Special Characters}
\DeclareTerm{special character}
\DeclareTerm{special characters}

In the \TeX\ language, certain \b{special characters} have special meanings that
cause them to be interpreted non-literally. That is, including such a character
in a \TeX\ source file will not yield that character in the output. Failure to
escape these characters will often cause compile failures.

A key aspect of the \TeX\ language is that in principle, \i{any character} can
have \i{any} special meaning, and these meanings can change during the
processing of a document. Fortunately, there is a conventional suite of
assignments that is nearly always active. Any person or computer program seeking
to understand \TeX\ code has virtually no choice but to assume that these
assignments are in effect unless context clues indicate otherwise.

The conventional special characters are:
\begin{itemize}
    \item \b{\texttt{\char`\\} (backslash)}, indicating the start of a control sequence
    \item \b{\texttt{\char`\{} (left curly brace)}, indicating the beginning of a group
    \item \b{\texttt{\char`\}} (right curly brace)}, indicating the ending of a group
    \item \b{\texttt{\char`\$} (dollar sign)}, indicating a math mode shift
    \item \b{\texttt{\char`\&} (ampersand)}, delimiting table columns
    \item \b{\texttt{\char`\#} (octothorp)}, indicating a macro parameter
    \item \b{\texttt{\char`\^} (caret)}, indicating a superscript
    \item \b{\texttt{\char`\_} (underscore)}, indicating a subscript
    \item \b{\texttt{\char`\~} (tilde)}, indicating a nonbreaking space
    \item \b{\texttt{\char`\%} (percent sign)}, indicating a comment
\end{itemize}
Space characters could also be considered to be somewhat special, since they
don’t behave totally literally: multiple consecutive spaces are collapsed, and
some spaces (e.g., ones after control sequences) fail to manifest in the final
output.

Likewise, ligatures can also cause some characters behave in a “special-like”
way. For instance, input of \texttt{``} is usually rendered as a
double left curly quote, “. Ligatures are generally intended to behave
intuitively, so if you type one unintentionally, the result will hopefully still
be what you want.

\section*{Escaping the standard special characters}

If you want one of the special characters to actually appear in your output
document, you must escape it in some fashion. In Tectonic, a reliable way to
escape these (and any other) characters is with the syntax \tex|\char`\X|, where
\texttt{X} is character in question. Some special characters, however, can be
escaped reliably with terser constructs. Recommended escape syntaxes are:
\begin{itemize}
    \item \b{\texttt{\char`\\} (backslash)}: \tex|\char`\\|
    \item \b{\texttt{\char`\{} (left curly brace)}: \tex`\{`
    \item \b{\texttt{\char`\}} (right curly brace)}: \tex`\}`
    \item \b{\texttt{\char`\$} (dollar sign)}: \tex`\$`
    \item \b{\texttt{\char`\&} (ampersand)}: \tex`\&`
    \item \b{\texttt{\char`\#} (octothorp)}: \tex`\#`
    \item \b{\texttt{\char`\^} (caret)}: \tex|\char`\^|
    \item \b{\texttt{\char`\_} (underscore)}: \tex`\_`
    \item \b{\texttt{\char`\~} (tilde)}: \tex|\char`\~|
    \item \b{\texttt{\char`\%} (percent sign)}: \tex`\%`
\end{itemize}


\Entry{at-escaping}{@-escaping}{@@-escaping}
\DeclareTerm{@-escaping}

The \b{@-escaping} syntax is a convention used by Tectonopedia to avoid writing
the \TeX\ \`special characters` in situations where it would be problematic to
process them.

For instance, Tectonpedia index terms can be rendered in two ways: as \TeX\
markup, where complex typography is possible using \TeX\ commands; and in “plain
style”, for when such typography is not possible. The term \TeX\ itself is a
good example: when possible, we wish to kern its ‘E’; but sometimes we must make
do with “TeX”. The latter is its plain style.

However, some plain-style terms contain \TeX\ \`special characters`, and it
becomes essentially intractable to express these terms reliably in \TeX\ source
with traditional escaping. The @-escaping syntax is an alternative that solves
this problem by avoiding \TeX\ special characters altogether.

The characters with @-escapes are:
\begin{itemize}
    \item \b{\texttt{\textbackslash} (backslash)}: \tex`@B` (\b{b}ackslash)
    \item \b{\texttt{\{} (left curly brace)}: \tex`@L` (\b{l}eft brace)
    \item \b{\texttt{\}} (right curly brace)}: \tex`@R` (\b{r}ight brace)
    \item \b{\texttt{\$} (dollar sign)}: \tex`@M` (\b{m}ath mode)
    \item \b{\texttt{\&} (ampersand)}: \tex`@A` (\b{a}mpersand)
    \item \b{\texttt{\#} (octothorp)}: \tex`@H` (\b{h}ash sign)
    \item \b{\texttt{\char`\^} (caret)}: \tex`@C` (\b{c}aret)
    \item \b{\texttt{\_} (underscore)}: \tex`@U` (\b{u}nderscore)
    \item \b{\texttt{\textasciitilde} (tilde)}: \tex`@N` (\b{n}on-breaking space)
    \item \b{\texttt{\%} (percent sign)}: \tex`@P` (\b{p}ercent sign)
    \item \b{\texttt{`} (backtick)}: \tex`@T` (\b{t}ick, back)
    \item \b{\texttt{@} (at-sign)}: \tex`@@` (an at-sign itself)
\end{itemize}
