\Entry{pedia_maybeVerbatimToks}{\string\pedia@maybeVerbatimToks}{\pedia@maybeVerbatimToks}
\DeclareTerm*{\string\pedia@maybeVerbatimToks}{@Bpedia@@maybeVerbatimToks}

The internal Tectonopedia command \b{\string\pedia@maybeVerbatimToks} references
a token list used during Tectonopedia verbatim processing.


\Entry{pedia_makeActiveWhitespace}{\string\pedia@makeActiveWhitespace}{\pedia@makeActiveWhitespace}
\DeclareTerm*{\string\pedia@makeActiveWhitespace}{@Bpedia@@makeActiveWhitespace}

The internal Tectonopedia command \b{\string\pedia@makeActiveWhitespace} sets
the category codes of whitespace be active.


\Entry{pedia_defineActiveWhitespace}{\string\pedia@defineActiveWhitespace}{\pedia@defineActiveWhitespace}
\DeclareTerm*{\string\pedia@defineActiveWhitespace}{@Bpedia@@defineActiveWhitespace}

The internal Tectonopedia command \b{\string\pedia@defineActiveWhitespace} sets
active whitespace characters to expand to the \tex`\ ` command.


\Entry{pediaScanVerbatim}{\string\pediaScanVerbatim}{\pediaScanVerbatim}
\DeclareTerm*{\string\pediaScanVerbatim}{@BpediaScanVerbatim}

The internal Tectonopedia command \b{\string\pediaScanVerbatim} causes the
immediately following balanced-braced text to be scanned in “verbatim” mode,
with control sequences and other special characters not having their usual \TeX\
meanings. The resulting list of character tokens is saved in the token list
\`@Bpedia@@maybeVerbatimToks`.

\section*{Example}

\begin{texdisp}
\pediaScanVerbatim{\hello$_$}
\makeatletter
\the\pedia@maybeVerbatimToks % expands to literal "\hello$_$"
\makeatother
\end{texdisp}


\Entry{pediaPassOneVerbatim}{\string\pediaPassOneVerbatim}{\pediaPassOneVerbatim}
\DeclareTerm*{\string\pediaPassOneVerbatim}{@BpediaPassOneVerbatim}

The internal Tectonopedia command \b{\string\pediaPassOneVerbatim} causes the
immediately following balanced-braced text to be scanned and saved in the token
list \`@Bpedia@@maybeVerbatimToks`.

On the first pass, the tokens are scanned in “verbatim” mode with
\`@BpediaScanVerbatim`, so the control sequences and other special characters do
not have their usual \TeX\ meaning. On the second pass, the text is scanned
without any special affordances, so control sequences and special characters are
handled normally.

\section*{Example}

\begin{texdisp}
\pediaPassOneVerbatim{6563 \AA}
\makeatletter
In pass one, the following will expand to a literal "6564 \AA". In pass
two, it will expand to "6563 Å".
\the\pedia@maybeVerbatimToks
\makeatother
\end{texdisp}

\section*{Remarks}

This command is needed for certain indexing operations. In the first pass, it is
necessary to extract verbatim \TeX\ sequences to index the marked-up textual
representations of various items. In the second pass, it is no longer necessary
to save the indexing information, and instead we need to actually show the
marked-up content.