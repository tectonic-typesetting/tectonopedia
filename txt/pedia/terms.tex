\entry{DeclareTerm}{\string\DeclareTerm}{\DeclareTerm}
\DeclareTerm*{\string\DeclareTerm}{@BDeclareTerm}

The Tectonopedia command \b{\string\DeclareTerm} declares a “term” usable for
wiki-style linking within a Tectonopedia collection.

\section*{Usage}

\begin{texdisp}
\DeclareTerm{TERM}
\DeclareTerm*{TEX-TERM}{PLAIN-TERM}
\end{texdisp}

The unstarred form of this command declares a term whose contents do not include
any special \TeX\ characters. The starred form must be used when the desired
typesetting of the term includes characters that have special meanings to the
\TeX\ language.

\section*{Example}

\begin{texdisp}
\DeclareTerm{gardening}
\DeclareTerm{\string\href}{@Bhref}

Many people enjoy \`gardening`.
\end{texdisp}

\section*{Remarks}

This command declares an entry in the \tex`terms` index whose name is the plain
form of the term value. The location associated with the index entry is the
current location at the time of the term’s definition.

This command does not expand to any text in the output document.

\section*{See Also}

\begin{itemize}
\item \`@B@T` — to link to a term defined using this command
\end{itemize}


\tduxEmit % ===================================================================

% backticks in the second argument here seem to cause serious parser issues!
% something to investigate.
\entry{(backtick)}{\char92\char96}{\`}
\DeclareTerm*{\string\`}{@B@T}

The Tectonopedia command \b{\string\`} (a single-letter control sequence
associated with a backtick) links to a term using wiki-style linking.

\section*{Usage}

\begin{texdisp}
\`TERM`
\end{texdisp}

The command has a special syntax. Its argument is terminated by the first
backtick character encountered by the \TeX\ parser after reading in the command.
The argument should be the “plain” textualization of a term declared using
\`@BDeclareTerm`.

\section*{Example}

\begin{texdisp}
\DeclareTerm{bicycling}

Some people enjoy \`bicycling`.
\end{texdisp}

\section*{Remarks}

This command expands to the \TeX\ marked-up form of the referenced term.

\section*{See Also}

\begin{itemize}
\item \`@BDeclareTerm` — to manually define a term for linking
\end{itemize}