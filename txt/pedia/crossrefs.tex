\entry{pediaLogRef}{\string\pediaLogRef}{\pediaLogRef}
\DeclareTerm*{\string\pediaLogRef}{@BpediaLogRef}

The internal Tectonopedia command \b{\string\pediaLogRef} logs a reference to an
index entry. These references are gathered during the first pass through the
inputs and validated during the cross-referencing stage. During the second pass
through the inputs, information about the referenced items will be made
available by the Tectonopedia driver.

\section*{Usage}

\begin{texdisp}
\pediaLogRef{INDEX}{ENTRY}{FLAGS}
\end{texdisp}

This logs a reference to the entry \tex`ENTRY` found in the specified index
\tex`INDEX`. The \tex`FLAGS` are characters indicating which kinds of
information about the entry are required.

\section*{Example}

\begin{texdisp}
\pediaLogRef{entry}{href}{l}
\end{texdisp}

\section*{Remarks}

This command does not expand to any text in the document.

The allowed flags are as follows:

% TODO definition list
\begin{itemize}
\item \tex`l` — the reference entry must have its location defined; the
    location is the URL at which the entry is found
\item \tex`t` — the reference entry must have its text defined; the text
    is a textual representation of the entry's name or identity
\end{itemize}


\tduxEmit % ===================================================================


\entry{pediaEnsureRefCS}{\string\pediaEnsureRefCS}{\pediaEnsureRefCS}
\DeclareTerm*{\string\pediaEnsureRefCS}{@BpediaEnsureRefCS}

The internal Tectonopedia command \b{\string\pediaEnsureRefCS} ensures that a
desired internal cross-referencing control sequence is defined. It is used to
ease compilation during the first pass through the inputs.

\section*{Usage}

\begin{texdisp}
\pediaEnsureRefCS{INDEX}{ENTRY}{DATATYPE}
\end{texdisp}

This ensures that there exists a control sequence whose name is
\tex`pedia resolve**INDEX**ENTRY**DATATYPE`, substituting in the arguments.
If this control sequence was not previously defined, it is defined to
expand to a question-mark character.

\section*{Example}

\begin{texdisp}
\pediaEnsureRefCS{entry}{href}{loc}
% Ensures that `\csname pedia resolve**entry**href**loc\endcsname` is defined
\end{texdisp}

\section*{Remarks}

This command does not expand to any text in the document itself.

During the first pass of Tectonopedia processing, cross-references haven't yet
been scanned, and none of the special control sequences used by this command are
expected to be defined. During the second pass, if the proper calls to
\`@BpediaLogRef` have been made, the relevant control sequences should have been
provided by the driver, and calls to this command should be effectively no-ops.

Standard \tex`DATATYPE` values include:

% TODO definition list
\begin{itemize}
\item \tex`loc` — for cross-references to entry locations (URLs)
\item \tex`text tex` — for cross-references to entry text in marked-up
    \TeX\ format
\item \tex`text plain` — for cross-references to entry text in
    plain Unicode
\end{itemize}


\tduxEmit % ===================================================================


\entry{pediaLinkRef}{\string\pediaLinkRef}{\pediaLinkRef}
\DeclareTerm*{\string\pediaLinkRef}{@BpediaLinkRef}

The Tectonopedia command \b{\string\pediaLinkRef} creates a hyperlink to an
entry in an index.

\section*{Usage}

\begin{texdisp}
\pediaLinkRef{INDEX}{ENTRY}
\end{texdisp}

This inserts an internal link whose text is the \TeX\ markup associated with
the named entry, and whose URL is its location.

\section*{Example}

\begin{texdisp}
For more information, see the \pediaLinkRef{entry}{href} page.
\end{texdisp}

\section*{Remarks}

This command expands to an \`@BhrefInternal` command whose URL and text are
derived from the indexing information associated with the named entry.

