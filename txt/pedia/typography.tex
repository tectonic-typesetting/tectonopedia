\Entry{b}{\string\b\ (Tectonopedia command)}{@Bb}
\DeclareTerm*{\string\b}{@Bb}

The Tectonopedia command \b{\string\b} is a short-hand for creating bold text.
It is identical to the \LaTeX\ command \tex`\textbf`.

\section*{Usage}

\begin{texdisp}
\b{TEXT}
\end{texdisp}

\section*{Example}

\begin{texdisp}
Use bold to \b{call attention} to an important term.
\end{texdisp}


\Entry{i}{\string\i\ (Tectonopedia command)}{@Bi}
\DeclareTerm*{\string\i}{@Bi}

The Tectonopedia command \b{\string\i} is a short-hand for creating italic,
emphasized text. It is identical to the \LaTeX\ command \tex`\emph`.

\section*{Usage}

\begin{texdisp}
\i{TEXT}
\end{texdisp}

\section*{Example}

\begin{texdisp}
Use italics for \b{emphasis}.
\end{texdisp}

\section*{Remarks}

The command name \tex`\e` might be a more natural abbreviation for \tex`emph`,
but it is used to reference encyclopedia entries instead.
