\Entry{Explainer}{\string\Explainer}{@BExplainer}
\DeclareTerm*{\string\Explainer}{@BExplainer}

The Tectonopedia command \b{\string\Explainer} is used to declare a new output
representing an “explainer” document.

\section*{Usage}

\begin{texdisp}
\Explainer{SLUG}{TEX-TITLE}{PLAIN-TITLE}
\end{texdisp}

The \tex`SLUG` is the unique URL path to this explainer. The explainer will be
located at the path \tex`/explain/{SLUG}`. Every explainer's slug must be unique.

The \tex`TEX-TITLE` is the title to be used for the explainer, including
\TeX/\LaTeX\ markup. This title will appear at the top of the resulting output
page.

The \tex`PLAIN-TITLE` is the explainer title in plain Unicode, using
\`@@-escaping` but not any \TeX/\LaTeX\ markup. It will be used in places such
as metadata records where \TeX/\LaTeX\ typesetting cannot be used.

\section*{Example}

\begin{texdisp}
\explainer{why-tex}{Why \TeX?}{Why TeX?}

Technology has come a long way since the 1970's ...
\end{texdisp}


\Entry{explain}{\string\explain}{@Bexplain}
\DeclareTerm*{\string\explain}{@Bexplain}

The Tectonopedia \b{\string\explain} command creates a link to the specified
explainer page.

\section*{Usage}

\begin{texdisp}
\explain{SLUG}
\end{texdisp}

The \tex`SLUG` is the slug of the explainer to reference.

In the output, this command expands to a hyperlink to the specified explainer whose
text is the entry title.

\section*{Example}

\begin{texdisp}
For more justification, see the \explain{why-tex} page.
\end{texdisp}
