\Entry{pediaTitle}{\string\pediaTitle}{\pediaTitle}
\DeclareTerm*{\string\pediaTitle}{@BpediaTitle}

The Tectonopedia command \b{\string\pediaTitle} renders a page title

\section*{Usage}

\begin{texdisp}
\pediaTitle{TITLE}
\end{texdisp}

\section*{Example}

\begin{texdisp}
\pediaTitle{Divio Documentation System}
\end{texdisp}

\section*{Remarks}

This command inserts its argument into the document, wrapped in a \tex`<div>`
tag with the \tex`pedia-pagetitle` CSS class.


\Entry{pediaBookName}{pediaBookName}{pediaBookName}
\DeclareTerm{pediaBookName}

The template variable \b{pediaBookName} specifies the name of the “book” to
which the current output page belongs.

\section*{Example}

\begin{texdisp}
\tduxSetTemplateVariable{pediaBookName}{Tectonopedia: Explainers}
\end{texdisp}

\section*{Remarks}

Pages within a Tectonopedia collection may be informally grouped into “books”.
The value of this variable is used to fill in the header line that appears at
the top of each output page.

The built-in Tectonpedia commands will manage the value of this variable for
you, so that you shouldn’t need to set it manually.
