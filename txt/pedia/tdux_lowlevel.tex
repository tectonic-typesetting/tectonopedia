\entry{tduxAddTemplate}{\string\tduxAddTemplate}{\tduxAddTemplate}
\DeclareTerm*{\string\tduxAddTemplate}{@BtduxAddTemplate}

The Tectonopedia command \b{\string\tduxAddTemplate} registers an HTML template
to be included in the Tera templating session.

\section*{Usage}

\begin{texdisp}
\tduxAddTemplate{TMPL-FILENAME}
\end{texdisp}

Here, \tex`TMPL-FILENAME` is a \TeX\ path to a Tera HTML template file.

\section*{Example}

\begin{texdisp}
\tduxAddTemplate{template.html}
\end{texdisp}

\section*{Remarks}

Templates can only be added before the first call to \`@BtduxEmit`. Template
names used here should correspond to the arguments given to
\`@BtduxSetupOutput`.

This command inserts a \tex`tdux:addTemplate` special.

\tduxEmit


\entry{tduxSetupOutput}{\string\tduxSetupOutput}{\tduxSetupOutput}
\DeclareTerm*{\string\tduxSetupOutput}{@BtduxSetupOutput}

The Tectonopedia command \b{\string\tduxSetupOutput} sets up the parameters of
the next output file.

\section*{Usage}

\begin{texdisp}
\tduxSetupOutput{TMPL-FILENAME}{OUTPUT-PATH}
\end{texdisp}

Here, \tex`TMPL-FILENAME` is the name of the HTML template to use. This must
have previously been registered using \`@BtduxAddTemplate`. \tex`OUTPUT-PATH` is
the path at which the output file will be created.

\section*{Example}

\begin{texdisp}
\tduxSetupOutput{template.html}{e/tduxSetupOutput/index.html}
\end{texdisp}

This would specify that the next output file will be created at the path
\tex`e/tduxSetupOutput/index.html`, using \tex`template.html` as the template.

\section*{Remarks}

Note that output files created through the Tectonic HTML system are created
directly on disk, and do not go through Tectonic’s virtualized I/O subsystem.

\tduxEmit


\entry{tduxEmit}{\string\tduxEmit}{\tduxEmit}
\DeclareTerm*{\string\tduxEmit}{@BtduxEmit}

The Tectonopedia command \b{\string\tduxEmit} causes the accumulated HTML
content to be emitted to disk, and then resets the emission state.

\section*{Usage}

\begin{texdisp}
\tduxEmit
\end{texdisp}

\section*{Example}

\begin{texdisp}
\tduxAddTemplate{template.html}

This is my about page!
\tduxSetupOutput{template.html}{pages/about/index.html}
\tduxEmit

\tduxSetupOutput{template.html}{pages/contact/index.html}
This is my contact page!
\tduxEmit
\end{texdisp}

This example creates two HTML outputs, \tex`pages/about/index.html` and
\tex`pages/contact/index.html`. Both use the template file \tex`template.html`.

\section*{Remarks}

The template to use must have previously been registered using
\`@BtduxAddTemplate`.

The directory structure needed to create the output path will be created if
needed. For instance, in the example above, the directories \tex`pages`,
\tex`pages/about`, and \tex`pages/contact` will be created if needed. Absolute
paths or relative paths that attempt to escape the HTML output toplevel are
illegal.

\tduxEmit


\entry{tduxSetTemplateVariable}{\string\tduxSetTemplateVariable}{\tduxSetTemplateVariable}
\DeclareTerm*{\string\tduxSetTemplateVariable}{@BtduxSetTemplateVariable}

The Tectonopedia command \b{\string\tduxSetTemplateVariable} assigns a value to
a variable, which can be used during the Tera templating process that creates
HTML output files. The value will persist until it is changed.

\section*{Usage}

\begin{texdisp}
\tduxSetTemplateVariable{NAME}{VALUE}
\end{texdisp}

The variable name \tex`NAME` may not contain whitespace.

\section*{Remarks}

Whether the variable contents are HTML-escaped is decided in the template,
through use of the Tera \tex`safe` directive.

\tduxEmit


\entry{tduxProvideFile}{\string\tduxProvideFile}{\tduxProvideFile}
\DeclareTerm*{\string\tduxProvideFile}{@BtduxProvideFile}

The Tectonopedia command \b{\string\tduxProvideFile} causes a source file to be
copied directly into the HTML output tree. This could be used, for instance, to
provide an image file.

\section*{Usage}

\begin{texdisp}
\tduxProvideFile{SOURCE-PATH}{DEST-PATH}
\end{texdisp}

Here, \tex`SOURCE-PATH` is the \TeX\ path of a source file, and \tex`DEST-PATH`
is the path of the file to be created in the output tree. \tex`SOURCE-PATH` may
not contain whitespace.

\section*{Remarks}

If the destination path contains directory separators, output directories are
created as needed.

\tduxEmit


\entry{tduxProvideSpecial}{\string\tduxProvideSpecial}{\tduxProvideSpecial}
\DeclareTerm*{\string\tduxProvideSpecial}{@BtduxProvideSpecial}

The Tectonopedia command \b{\string\tduxProvideSpecial} causes one or more
special, internally-generated output files to be created in an HTML output tree.

\section*{Usage}

\begin{texdisp}
\tduxProvideSpecial{KIND}{DEST-PATH}
\end{texdisp}

Here, \tex`KIND` is the kind of special output to create. Currently the only
supported value is \tex`font-css`.

\section*{Example}

\begin{texdisp}
\tduxProvideSpecial{font-css}{tdux-fonts.css}
\end{texdisp}

\section*{Remarks}

Using the \tex`font-css` “kind” causes the specified output file to be filled
with CSS code that declares information about all of the fonts used during the
\TeX\ processing stage, including special variant files created implicitly.
