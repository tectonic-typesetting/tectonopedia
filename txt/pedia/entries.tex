\Entry{entry-command}{\string\Entry}{@BEntry}
\DeclareTerm*{\string\Entry}{@BEntry}

The Tectonopedia command \b{\string\Entry} is used to declare a new output
representing an entry in the reference section of the collection.

\section*{Usage}

\begin{texdisp}
\Entry{SLUG}{TEX-TITLE}{PLAIN-TITLE}
\end{texdisp}

The \tex`SLUG` is the unique URL path to this entry. The entry will be located
at the path \tex`/e/{SLUG}`. Every entry's slug must be unique.

The \tex`TEX-TITLE` is the title to be used for the entry, including
\TeX/\LaTeX\ markup. This title will appear at the top of the resulting output
page.

The \tex`PLAIN-TITLE` is the entry title in plain Unicode, without any
\TeX/\LaTeX\ markup. It is parsed in verbatim mode, so that control sequences
and other special characters do not have their usual \TeX\ meanings. It will be
used in places such as metadata records where \TeX/\LaTeX\ typesetting cannot be
used.

\section*{Example}

\begin{texdisp}
\Entry{tduxAddTemplate}{\string\tduxAddTemplate}{@BtduxAddTemplate}

The Tectonopedia command \b{\string\tduxAddTemplate} registers an HTML template ...
\end{texdisp}


\Entry{e}{\string\e}{@Be}
\DeclareTerm{@Be}

The Tectonopedia \b{\string\e} command creates a link to the specified
encyclopedia reference entry.

\section*{Usage}

\begin{texdisp}
\e{SLUG}
\end{texdisp}

The \tex`SLUG` is the slug of the entry to reference.

In the output, this command expands to a hyperlink to the specified entry whose
text is the entry title.

\section*{Example}

\begin{texdisp}
The Tectonopedia structure is influenced by the
\e{divio-documentation-system} design.
\end{texdisp}

\section*{Remarks}

In many cases, it may be more natural to use wiki-style linking using the
\`@BDeclareTerm` and \`@B@T` commands. In effect, wiki-style links offer better
control over the link text that appears in the output document. When referencing
an entry with \tex`\e`, the link text will always be the entry title. With
wiki-style linking, on the other hand, you can achieve similar levels of
convenience with more flexibility.
