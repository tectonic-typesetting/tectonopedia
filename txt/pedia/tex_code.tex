\Entry{tex}{\string\tex}{@Btex}
\DeclareTerm*{\string\tex}{@Btex}

The Tectonopedia command \b{\string\tex} is used to render \TeX/\LaTeX\ code
that appears inline in body text.

\section*{Usage}

\begin{texdisp}
\tex`TEX-CODE`
\end{texdisp}

The \tex`TEX-CODE` is scanned in verbatim mode, so that any control sequences or
other characters special to \TeX\ do not have their usual effects.

Any character may follow the \tex`\tex` control sequence; the following tokens
are scanned in verbatim mode until the next occurrence of the delimiter
character. In Tectonopedia, the standard choice for the delimiter is the
backtick character (`).

In the output, the \tex`TEX-CODE` is rendered in a monospace font.

\section*{Example}

\begin{texdisp}
To create bold text in the Tectonopedia, write \tex`\b{text}`.
\end{texdisp}

\section*{Remarks}

This command is currently implemented with the \tex`\Verb` control sequence from
the \tex`fancyvrb`. In the future, this command may add features such as syntax
highlighting.

\section*{See Also}

\begin{itemize}
\item The \`texdisp` environment — for display-style \TeX/\LaTeX\ code
\end{itemize}


\Entry{texdisp}{texdisp}{texdisp}
\DeclareTerm{texdisp}

The \b{texdisp} environment provided by Tectonopedia is used to render
\TeX/\LaTeX\ code in a display format.

\section*{Usage}

% I presume that we can't nest texdisp environments, so use Verbatim ...
\begin{Verbatim}
\begin{texdisp}
VERBATIM-DISPLAY-CONTENT
\end{texdisp}
\end{Verbatim}

The \tex`VERBATIM-DISPLAY-CONTENT` is scanned in verbatim mode, so that any
control sequences or other characters special to \TeX\ do not have their usual
effects.

In the output, the verbatim display content is rendered in a monospace font and
in a display presentation.

\section*{Example}

\begin{Verbatim}
To declare a new term for wiki-style linking, write:

\begin{texdisp}
\DeclareTerm{my term}
\end{texdisp}
\end{Verbatim}
    
\section*{Remarks}

This environment is equivalent to the \tex`Verbatim` environment provided by the
\tex`fancyvrb` package. In the future, it command may add features such as
syntax highlighting.

\section*{See Also}

\begin{itemize}
\item \`@Btex` — for inlne \TeX/\LaTeX\ code
\end{itemize}
