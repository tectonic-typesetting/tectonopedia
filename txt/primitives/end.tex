\Entry{end}{\string\end\ (\TeX\ primitive)}{@Bend (TeX primitive)}
% We can't use "\tex`\end`" in the TeX form of the term, because it
% is tokenized during the macro definition, so the macro expansion works out to
% `(\tex cseq)(backtick)(\end cseq)(backtick)` -- the \tex command doesn't have
% the chance to get the tokenizer to handle `\end` specially. I am not sure if
% there is a way to solve this either.
\DeclareTerm*{the \string\end\ primitive}{the @Bend primitive}

\i{Not to be confused with LaTeX's \string\end\ command.}

The \`TeX primitive` \tex`\end` finishes processing of the document.

In \LaTeX, the \tex`\end` control sequence is redefined to be a command that
closes the current environment. The \tex`\end{document}` version of this command
is similar in spirit to the primitive, causing \LaTeX\ to wrap up processing of
the document. This process finishes by invoking the \tex`\end` primitive, which
is directly accessible in \LaTeX\ through the control sequence \tex`\@@end`.
There should almost never be a reason to invoke it directly in user code,
however.

When the \tex`\end` primitive is invoked, processing only terminates if the main
vertical list is empty and the \tex`\deadcycles` parameter is zero. If this is
not the case, the engine inserts content equivalent to %
\tex`\line{} \vfill \penalty-'10000000000` into the main vertical list and then
prepares to reread the \tex`\end` token again. This causes the output routine to
be invoked and should eventually cause the main vertical list to be completely
emptied. However, it is possible to construct an output routine that will never
allow the criteria above to be met, meaning that document processing will never
exit.

\section*{See Also}

\begin{itemize}
\item \e{dump}, used to end \`format` definitions
\end{itemize}
