% Copyright 2022-2023 the Tectonic Project
% Licensed under the MIT License
%
% Encyclopedia "entries": individual reference items.
%
% \entry{SLUG}{TEX-TITLE}{PLAIN-TITLE}
%  Define an entry
%
%  Because the entry titles need to be parsed as verbatim code for
%  indexing, the actual implementation of this command is split.
\makeatletter
\newtoks\pedia@titletmp
%
\newcommand{\entry}[1]{%
  % save the slug:
  \def\tmp@b{#1}%

  % This is the stuff we can do with just the slug:
  \tduxSetupOutput{template.html}{e/#1/index.html}
  \def\pediaRelTop{../../}
  \immediate\write\pediaIndex{\string\output{e/#1/index.html}}
  \immediate\write\pediaIndex{\string\idef{entries}{#1}{}}
  % This parses the second argument (the TeX title), places it in
  % \pedia@maybeVerbatimToks, and then evaluates \entry@tailA
  \pediaPassOneVerbatim\entry@tailA
}
\newcommand{\entry@tailA}{%
  % Render the entry title. On pass 1 it will have been parsed verbatim,
  % so any TeX constructs will appear literally. This is fine since we don't
  % actually use the pass 1 HTML!
  \pediaTitle{\the\pedia@maybeVerbatimToks}%
  %
  \pedia@titletmp=\pedia@maybeVerbatimToks
  \pediaScanVerbatim\entry@tailB
}
\newcommand{\entry@tailB}{%
  % The point of all of the verbatim gymnastics: we can emit the "text" of the
  % entry as the literal TeX code that the user provided, rather than whatever
  % that code expands to.
  \immediate\write\pediaIndex{\string\itext{entries}{\tmp@b}{\the\pedia@titletmp}{\the\pedia@maybeVerbatimToks}}

  % Finally we can also set the page title
  \tduxSetTemplateVariable{pediaTitle}{\the\pedia@maybeVerbatimToks}
  \tduxSetTemplateVariable{pediaBookName}{Tectonopedia: The Reference}
}
\makeatother
%
% \e{ENTRY}
%  Create an internal link to the specified encyclopedia entry.
\newcommand{\e}[1]{%
  \pediaLinkRef{entries}{#1}%
}
