% Copyright 2022-2023 the Tectonic Project
% Licensed under the MIT License
%
% Articles in the "explainer" category.  Documentation in
% `~/txt/pedia/explainers.tex`.
%
%  TODO: massive duplication with other page types!!!!
\makeatletter
%
\newcommand{\Explainer}[1]{%
  % finish a previous output, if needed:
  \pedia@maybeEmit

  % save the slug:
  \def\tmp@b{#1}%

  % This is the stuff we can do with just the slug:
  \tduxSetupOutput{template.html}{explain/#1/index.html}
  \@pedia@emitNeededtrue
  \def\pediaRelTop{../../}
  \immediate\write\pediaIndex{\string\output{explain/#1/index.html}}
  \immediate\write\pediaIndex{\string\idef{explainers}{#1}{}}
  % This parses the second argument (the TeX title), places it in
  % \pedia@maybeVerbatimToks, and then evaluates \explainer@tailA
  \pediaPassOneVerbatim\explainer@tailA
}
\newcommand{\explainer@tailA}{%
  % Render the explainer title. On pass 1 it will have been parsed verbatim,
  % so any TeX constructs will appear literally. This is fine since we don't
  % actually use the pass 1 HTML!
  \pediaTitle{\the\pedia@maybeVerbatimToks}%
  %
  \pedia@titletmp=\pedia@maybeVerbatimToks
  \pediaScanVerbatim\explainer@tailB
}
\newcommand{\explainer@tailB}{%
  % The point of all of the verbatim gymnastics: we can emit the "text" of the
  % explainer as the literal TeX code that the user provided, rather than whatever
  % that code expands to.
  \immediate\write\pediaIndex{\string\itext{explainers}{\tmp@b}{\the\pedia@titletmp}{\the\pedia@maybeVerbatimToks}}

  % Finally we can also set the page title
  \tduxSetTemplateVariable{pediaTitle}{\the\pedia@maybeVerbatimToks}
  \tduxSetTemplateVariable{pediaBookName}{Tectonopedia: Explainers}
}
\makeatother
%
% \explain{ENTRY}
%  Create an internal link to an explainer, identified by its slug.
\newcommand{\explain}[1]{%
  \pediaLinkRef{explainers}{#1}%
}
