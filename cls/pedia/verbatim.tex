% Copyright 2022-2023 the Tectonic Project
% Licensed under the MIT License
%
% Setting up verbatim processing. Documentation in `~/txt/pedia/verbatim.tex`.
%
\RequirePackage{fancyvrb}
%
% For some things, during pass one we want to scan text as verbatim, but in pass
% 2 we want to actually evaluate the tokens. The construct
% \pediaPassOneVerbatim{AFTER}{TEXT} helps with that. It scans the braced
% expression TEXT into the toklist \pedia@maybeVerbatimToks, and then expands
% out the argument AFTER. In pass 1, TEXT will be scanned in "verbatim mode",
% while otherwise it will be scanned with the usual TeX tokenization rules. Due
% to the way that TeX's tokenization works, this is a super fragile command, and
% it has to appear as the last command of any macros that you define to use it.
\makeatletter
\newtoks{\pedia@maybeVerbatimToks}
\def\pedia@makeActiveWhitespace{%
  \catcode`\ =\active%
  \catcode`\^^I=\active%
}%
\begingroup%
  % for our purposes, tabs are just spaces
  \pedia@makeActiveWhitespace%
  \gdef\pedia@defineActiveWhitespace{\def {\ }\def^^I{\ }}%
\endgroup
\def\pediaScanVerbatim#1{%
  \begingroup
  % This setup derived from fancyvrb.sty rather than the plain TeX macros.
  % Recall that when this macro is expanded, everything here will already have
  % been read into a token a list, so changing the catcode of `\` doesn't
  % break our syntax here.
  %
  % Note that we are *not* changing the meaning of {} here, because we need
  % balanced delimiters for our toklist scanning. TODO: offer a mode that uses
  % different delimiters so that expressions with unbalanced braces can be
  % read. We can't do the `\verb||` trick because that requires that tokens
  % we're scanning are being expanded as we go, while we need to be saving
  % them into a toklist.
  \catcode`\\=12
  \catcode`\$=12
  \catcode`\&=12
  \catcode`\#=12
  \catcode`\%=12
  \catcode`\~=12
  \catcode`\_=12
  \catcode`\^=12
  % Activate the mono font, which doesn't have TeX ligatures enabled. When we
  % are doing this verbatim-mode scanning, that's what we want. fancyvrb uses
  % \@noligs, but that has a problem because it defines ligature-prone
  % characters to be active, which means that if/when we try to expand the
  % resulting tokens we get undefined-control-sequence errors, because the
  % active meanings of those characters are only defined within this group.
  \ttfamily
  %
  \pedia@makeActiveWhitespace
  \pedia@defineActiveWhitespace
  \def\tmp@a{\pedia@afterScanVerbatim #1}%
  \afterassignment\tmp@a
  \global\pedia@maybeVerbatimToks=%
}%
% This macro is expanded after the verbatim assignment is finished; it completes
% the handling of the verbatim text.
\def\pedia@afterScanVerbatim{%
  \endgroup
}%
\ifpassone
  \let\pediaPassOneVerbatim=\pediaScanVerbatim
\else
  % When we're not in pass 1, the tokens are scanned with normal TeX rules so
  % that control sequences are parsed. TODO: handle any special delimiters
  % provided above.
  \def\pediaPassOneVerbatim#1{%
    \def\tmp@a{#1}%
    \afterassignment\tmp@a
    \global\pedia@maybeVerbatimToks=%
  }%
\fi
\makeatother
